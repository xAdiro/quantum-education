\documentclass{SGGW-thesis}

\INZYNIERSKAtrue
\WZIMtrue

\title{Wykorzystanie technologii webowych i języka Python do stworzenia aplikacji edukacyjnej z mechaniki kwantowej}
\Etitle{Utilizing web technologies and Python language to create quantum physics educational application}

\author{Adrian Rostek}
\album{205860}
\thesis{Praca dyplomowa na kierunku:}
\course{Informatyka}
\promotor{dr Andrzeja Zembrzuskiego}
\pworkplace{Instytut Informatyki Technicznej\\Katedra Sztucznej Inteligencji}

\usepackage{hyperref}

\begin{document}
\maketitle
\statementpage
\abstractpage
{Wykorzystanie technologii webowych i języka Python do stworzenia aplikacji edukacyjnej z mechaniki kwantowej}
{...Treść streszczenia...}
{Edukacja, Fizyka kwantowa, Funkcja falowa, Wizualizacja}
{Utilizing web technologies and Python language to create quantum physics educational application}
{...Treść angielskiego streszczenia...}
{Education, Quantum physics, Wave function, Visualization}


{
  % Spis treści może być złożony z pojedynczą interlinią, np. jeśli jedna linia wychodzi na następną stronę.
  % W przeciwnym razie spis treści wstawić bez powyższego rozkazu i klamry.
  \doublespacing
  \tableofcontents
}

\startchapterfromoddpage % niezależnie od długości spisu treści pierwszy rozdział zacznie się na nieparzystej stronie

\chapter{Wstęp}
wstep...


	\section{Cel i zakres pracy}
	cel...\cite{JS}


\begin{thebibliography}{9}
	\bibitem{fiz atom}
	M.R. Wehre, H.A. Enge, J.A. Richards,
	\textit{Wstęp do fizyki atomowej}, 
	Państwowe Wydawnictwo naukowe, Warszawa 1983
	
	\bibitem{JS}
	David Flanagan, 
	\textit{JavaScript: The Definitive Guide. Master the World's Most-Used Programming Language. 7th Edition}, 
	O'Reilly Media, 2020
	
	
\end{thebibliography}

\beforelastpage

\end{document}
