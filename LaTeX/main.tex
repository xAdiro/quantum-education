\documentclass{SGGW-thesis}
\usepackage{xcolor}

\INZYNIERSKAtrue
\WZIMtrue

\title{Wykorzystanie technologii webowych i języka Python do stworzenia aplikacji edukacyjnej z mechaniki kwantowej}
\Etitle{Utilizing web technologies and Python language to create quantum physics educational application}

\author{Adrian Rostek}
\album{205860}
\thesis{Praca dyplomowa na kierunku:}
\course{Informatyka}
\promotor{dr Andrzeja Zembrzuskiego}
\pworkplace{Instytut Informatyki Technicznej\\Katedra Sztucznej Inteligencji}

\usepackage{hyperref}

\begin{document}
\maketitle
\statementpage
\abstractpage
{Wykorzystanie technologii webowych i języka Python do stworzenia aplikacji edukacyjnej z mechaniki kwantowej}
{...Treść streszczenia...}
{Edukacja, Fizyka kwantowa, Funkcja falowa, Wizualizacja}
{Utilizing web technologies and Python language to create quantum physics educational application}
{...Treść angielskiego streszczenia...}
{Education, Quantum physics, Wave function, Visualization}


{
  % Spis treści może być złożony z pojedynczą interlinią, np. jeśli jedna linia wychodzi na następną stronę.
  % W przeciwnym razie spis treści wstawić bez powyższego rozkazu i klamry.
  \doublespacing
  \tableofcontents
}

\startchapterfromoddpage % niezależnie od długości spisu treści pierwszy rozdział zacznie się na nieparzystej stronie

\chapter{Wstęp}
Stworzenie teorii mechaniki kwantowej w 1925 r.[przypis?] okazało się podstawą dzisiejszej cywilizacji[przypis]. Zawdzięczamy jej m.in. tranzystory tworzące komputery, mikroskopy tunelowe o niebywałej precyzji oraz reaktory jądrowe, bez których ciężko wobrazić sobie dzisiejszą energetykę[przypis]. Mimo to jest to bardzo nieintuicyjny i przez większośc ludzi niezrozumiały dział fizyki[przypis]. Na temat ten napisane zostały liczne publikacje[przypis], jednak profesjonalny język i matematyka wyższa mogą sprawić dużo trudności w zrozumieniu nawet podstawowych konceptów tej teorii.


	\section{Cel i motywacja pracy}
	Celem pracy jest stworzenie aplikacji, która ma ułatwić naukę zagadnień z zakresu mechaniki kwantowej. Zagadnienia przedstawiane są w interaktywny celem podtrzymania uwagi i zainteresowania tym nietrywialnym tematem. Osiągnięte to zostało poprzez wykorzystanie licznych symulacji, na których efekt końcowy bezpośredni wpływ ma użytkownik, jednocześnie stosując samouczek, który te efekty odpowiednio tłumaczy.
	
	Osobiście temat mechaniki kwantowej uważam za niesamowicie ciekawy, więc napisanie tej pracy motywowane jest chęcią poszerzenia swojej wiedzy w tym obszarze, jak i zastosowaniu nabytej wiedzy informatycznej w stworzeniu praktycznego narzędzia. Za interesujące również uważam symulację funkcji falowej w przeciwieństwie do przypatrywania się statycznym jej wykresom na papierze czy w plikach pdf. Aplikacja kierowana jest do osób chcących nauczyć się wstępnych zagadnień mechaniki kwantowej, jednak bez konieczności sięgania po profesjonalną literaturę. Do pełnego zrozumienia wszystkich zagadnień potrzebna jest znajomość matematyki wyższej, jednak nawet bez tej wiedzy użytkownik może wynieść z aplikacji dużo nowych informacji. Może ona być więc użyteczna zarówno dla osób nie będących ściśle związanych z naukami matematycznymi i fizycznymi, jak i \textcolor{red}{studentów kierunków fizycznych?}.
	\section{Tematyka i struktura pracy}
	Aplikacja przytacza kontekst historyczny dziedziny fizyki jaką jest mechanika kwantowa jak i tłumaczy falowo korpuskularną naturę cząstek. Główna częśc jednak skupia się na typowych rozwiązaniach równania Schrödingera niezależnego od czasu, a dokładniej:
	\begin{itemize}
	\item Cząstki swobodnej
	\item Nieskończonej studni potencjału
	\item Skończonej studni potencjału
	\item Progu potencjału
	\item Bariery potencjału
	\end{itemize}
	
	Jest to często stosowana kolejność wprowadzania do tych zagadnień[przypis], ponieważ każdy kolejny przypadek bazuje na poprzednim, wprowadzając jednak stopniowo coraz to nowsze elementy. Wytłumaczone są też zjawiska tunelowe i skwantowanych stanów energetycznych jako konsekwencja dotychczas przyswojonych zagadnień.
	
	
\chapter{Wykorzystane technologie}
	\section{Popularne technologie webowe - HTML, CSS i TypeScript}
	\section{Biblioteki Chart.js i MathJax}
	\section{Język Python}
	\section{Framework Tauri}
	Była opcja WebAssembly ale jest niedorpacowane(napisać szczegóły)
	+ vite + npm
	
\chapter{Podstawy teorytyczne}
	\section{Falowa natura materii}
	\section{Równanie Schrodingera}
	\section{Znajdowanie funkcji falowej}
	
\chapter{Budowa i struktura aplikacji}
	\section{Struktura aplikacji}
	domyślne ustawienie Tauri, zasoby, pliki html, css, typescript, python
	\section{Interfejs}
	html + css, BEM, struktura stron
	\section{Typescript i manipulacja DOM}
	\section{Obliczenia fizyczne w Pythonie}
	\section{Interfejs pomiędzy TypeScriptem i Pythonem}
	\section{Wdrożenie i dystrybucja aplikacji}
	o budowie aplikacji, pakowanych zasobach, interpreterze pythona, platformach
	\section{Problemy, ograniczenia i możliwości rozwoju}

\chapter{Interfejs aplikacji}
	\section{Ekran główny}
	\section{Interaktywna wizualizacja}
	\section{Transkrypcja}
	
\chapter{Podsumowanie i wnioski}


\begin{thebibliography}{9}
	\bibitem{fiz atom}
	M.R. Wehre, H.A. Enge, J.A. Richards,
	\textit{Wstęp do fizyki atomowej}, 
	Państwowe Wydawnictwo naukowe, Warszawa 1983
	
	\bibitem{JS}
	David Flanagan, 
	\textit{JavaScript: The Definitive Guide. Master the World's Most-Used Programming Language. 7th Edition}, 
	O'Reilly Media, 2020
	
	
\end{thebibliography}

\beforelastpage

\end{document}
