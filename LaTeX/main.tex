\documentclass{SGGW-thesis}
\usepackage{xcolor}

\INZYNIERSKAtrue
\WZIMtrue

\title{Wykorzystanie technologii webowych i języka Python do stworzenia aplikacji edukacyjnej z mechaniki kwantowej}
\Etitle{Utilizing web technologies and Python language to create quantum physics educational application}

\author{Adrian Rostek}
\album{205860}
\thesis{Praca dyplomowa na kierunku:}
\course{Informatyka}
\promotor{dr Andrzeja Zembrzuskiego}
\pworkplace{Instytut Informatyki Technicznej\\Katedra Sztucznej Inteligencji}

\usepackage{hyperref}

\begin{document}
\maketitle
\statementpage
\abstractpage
{Wykorzystanie technologii webowych i języka Python do stworzenia aplikacji edukacyjnej z mechaniki kwantowej}
{...Treść streszczenia...}
{Edukacja, Fizyka kwantowa, Funkcja falowa, Wizualizacja}
{Utilizing web technologies and Python language to create quantum physics educational application}
{...Treść angielskiego streszczenia...}
{Education, Quantum physics, Wave function, Visualization}


{
  % Spis treści może być złożony z pojedynczą interlinią, np. jeśli jedna linia wychodzi na następną stronę.
  % W przeciwnym razie spis treści wstawić bez powyższego rozkazu i klamry.
  \doublespacing
  \tableofcontents
}

\startchapterfromoddpage % niezależnie od długości spisu treści pierwszy rozdział zacznie się na nieparzystej stronie

\chapter{Wstęp}
wstep...


	\section{Cel i zakres pracy}
	Celem pracy jest stworzenie aplikacji, która ma ułatwić naukę zagadnień z zakresu mechaniki kwantowej. Zagadnienia przedstawiane są w interaktywny sposób tak, aby podtrzymać uwagę i zainteresowanie tym nietrywialnym tematem.
	
	Aplikacja kierowana jest do osób chcących nauczyć się wstępnych zagadnień mechaniki kwantowej, jednak bez konieczności sięgania po profesjonalną literaturę. Do pełnego zrozumienia wszystkich zagadnień potrzebna jest znajomość matematyki wyższej, jednak nawet bez tej wiedzy użytkownik może wynieść z aplikacji dużo nowych informacji. Może ona być więc użyteczna zarówno dla osób nie będących ściśle związanych z naukami matematycznymi i fizycznymi, jak i \textcolor{red}{studentów kierunków fizycznych?}.
	Interaktywność
	\section{Tematyka i struktura pracy}
	
\chapter{Wykorzystane technologie}
	\section{Popularne technologie webowe - HTML, CSS i TypeScript}
	\section{Biblioteki Chart.js i MathJax}
	\section{Język Python}
	\section{Framework Tauri}
	Była opcja WebAssembly ale jest niedorpacowane(napisać szczegóły)
	+ vite + npm
	
\chapter{Podstawy teorytyczne}
	\section{Falowa natura materii}
	\section{Równanie Schrodingera}
	\section{Znajdowanie funkcji falowej}
	
\chapter{Budowa i struktura aplikacji}
	\section{Struktura aplikacji}
	domyślne ustawienie Tauri, zasoby, pliki html, css, typescript, python
	\section{Interfejs}
	html + css, BEM, struktura stron
	\section{Typescript i manipulacja DOM}
	\section{Obliczenia fizyczne w Pythonie}
	\section{Interfejs pomiędzy TypeScriptem i Pythonem}
	\section{Wdrożenie i dystrybucja aplikacji}
	o budowie aplikacji, pakowanych zasobach, interpreterze pythona, platformach
	\section{Problemy, ograniczenia i możliwości rozwoju}

\chapter{Interfejs aplikacji}
	\section{Ekran główny}
	\section{Interaktywna wizualizacja}
	\section{Transkrypcja}
	
\chapter{Podsumowanie i wnioski}


\begin{thebibliography}{9}
	\bibitem{fiz atom}
	M.R. Wehre, H.A. Enge, J.A. Richards,
	\textit{Wstęp do fizyki atomowej}, 
	Państwowe Wydawnictwo naukowe, Warszawa 1983
	
	\bibitem{JS}
	David Flanagan, 
	\textit{JavaScript: The Definitive Guide. Master the World's Most-Used Programming Language. 7th Edition}, 
	O'Reilly Media, 2020
	
	
\end{thebibliography}

\beforelastpage

\end{document}
